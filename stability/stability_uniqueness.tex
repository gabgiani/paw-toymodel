\documentclass[11pt,a4paper]{article}

\usepackage[utf8]{inputenc}
\usepackage[T1]{fontenc}
\usepackage{amsmath,amssymb,amsthm,mathtools}
\usepackage{geometry}
\usepackage{hyperref}
\usepackage{enumitem}
\usepackage{xcolor}

\geometry{margin=2.5cm}

\hypersetup{
  colorlinks=true,
  linkcolor=blue!70!black,
  citecolor=blue!70!black,
  urlcolor=blue!70!black
}

% Theorem environments
\theoremstyle{plain}
\newtheorem{theorem}{Theorem}[section]
\newtheorem{proposition}[theorem]{Proposition}
\newtheorem{lemma}[theorem]{Lemma}
\newtheorem{corollary}[theorem]{Corollary}

\theoremstyle{definition}
\newtheorem{definition}[theorem]{Definition}

\theoremstyle{remark}
\newtheorem{remark}[theorem]{Remark}

% Commands
\newcommand{\hilb}{\mathcal{H}}
\newcommand{\End}{\mathrm{End}}
\newcommand{\Tr}{\mathrm{Tr}}
\newcommand{\Ad}{\mathrm{Ad}}
\newcommand{\Norm}{\mathrm{Norm}}
\newcommand{\HS}{\mathrm{HS}}
\newcommand{\id}{\mathbb{I}}
\newcommand{\CC}{\mathbb{C}}
\newcommand{\RR}{\mathbb{R}}
\newcommand{\calA}{\mathcal{A}}
\newcommand{\calB}{\mathcal{B}}
\newcommand{\calF}{\mathcal{F}}
\newcommand{\calG}{\mathcal{G}}
\newcommand{\calK}{\mathcal{K}}
\newcommand{\calL}{\mathcal{L}}

\title{\textbf{Stability and Uniqueness of Tensor Product Structures\\[4pt] under Hamiltonian Dynamics}\\[8pt]
\large A Mathematical Framework for Dynamically Selected Factorizations\\in the Page--Wootters Formalism}

\author{Gabriel Giani}

\date{February 2026}

\begin{document}
\maketitle

\begin{abstract}
We develop a self-contained mathematical framework establishing that
the tensor product structure (TPS) $\hilb_S\otimes\hilb_E$ 
presupposed by the partial trace operation in the Page--Wootters formalism
is not an arbitrary choice but is dynamically selected by the Hamiltonian.
Working in finite dimensions with Hilbert--Schmidt geometry, we prove:
(1)~a quadratic bound on the growth of mutual information for initially product states,
controlled by the interaction component of the Hamiltonian;
(2)~that any two bipartitions satisfying a stability condition must have
nearly parallel local subspaces;
(3)~that the unitary connecting two such bipartitions is necessarily almost local;
(4)~a variational principle identifying the physical factorization as the minimizer
of interaction strength; and
(5)~that the effective dynamics of the reduced state is almost unitary, with
dissipation controlled by the interaction strength.
All results are stated with explicit hypotheses and constants.
\end{abstract}

\medskip
\noindent\textbf{Keywords:} tensor product structure, Page--Wootters formalism,
Hilbert--Schmidt geometry, quantum factorization, mutual information,
stability, uniqueness.

%=============================================================================
\section{Introduction and Motivation}
%=============================================================================

The Page--Wootters mechanism produces conditional dynamics from a globally
static state $H_{\mathrm{tot}}|\Psi\rangle=0$ via the projection
\begin{equation}\label{eq:pw-formula}
  \rho_S(t) \;=\;
  \frac{\Tr_E\!\bigl[\langle t|_C\,\rho_{CSE}\,|t\rangle_C\bigr]}{p(t)},
  \qquad
  p(t) = \Tr_{SE}\!\bigl[\langle t|_C\,\rho_{CSE}\,|t\rangle_C\bigr].
\end{equation}
This formula presupposes a factorization $\hilb_{SE}\cong\hilb_S\otimes\hilb_E$
in which the partial trace $\Tr_E$ is performed~\cite{PageWootters1983,Wootters1984}.
A natural question arises: \emph{is this factorization unique, or could a different
choice of subsystems lead to different physical predictions?}

We provide a rigorous mathematical answer, working entirely within
finite-dimensional quantum mechanics and Hilbert--Schmidt (HS) geometry.
No philosophical assumptions about observers or measurement are invoked.

\medskip
\noindent\textbf{Conventions.}
Throughout, $\hilb=\hilb_S\otimes\hilb_E$ with $d_S=\dim\hilb_S\ge 2$,
$d_E=\dim\hilb_E\ge 2$, both finite.
The Hilbert--Schmidt inner product on $\End(\hilb)$ is
$\langle X,Y\rangle:=\Tr(X^\dagger Y)$ with norm $\|X\|_2:=\sqrt{\langle X,X\rangle}$.
For superoperators acting on $\End(\hilb)$, $\|T\|_{2\to 2}:=\sup_{\|X\|_2=1}\|T(X)\|_2$
denotes the induced norm. The operator norm is $\|X\|:=\sup_{\|v\|=1}\|Xv\|$.

%=============================================================================
\section{Setup: Bipartitions and the HS Decomposition}
\label{sec:setup}
%=============================================================================

\begin{definition}[Local subspace]\label{def:local}
Given a bipartition $S|E$, define the \emph{local algebras}
\[
  \calA := \End(\hilb_S)\otimes I_E, \qquad
  \calB := I_S\otimes \End(\hilb_E),
\]
their traceless parts
\[
  \calA_0 := \{A\otimes I : \Tr(A)=0\}, \qquad
  \calB_0 := \{I\otimes B : \Tr(B)=0\},
\]
and the \emph{local subspace}
\[
  \calL_0 := \calA_0 \oplus \calB_0
  \quad\subset\; \End_0(\hilb) := \{X\in\End(\hilb):\Tr(X)=0\}.
\]
\end{definition}

\begin{lemma}[Orthogonality]\label{lem:ortho}
$\calA_0\perp\calB_0$ in the HS inner product, and the decomposition
$\calL_0=\calA_0\oplus\calB_0$ is orthogonal and direct, with
$\dim\calA_0=d_S^2-1$, $\dim\calB_0=d_E^2-1$.
\end{lemma}

\begin{proof}
For $A\otimes I\in\calA_0$ and $I\otimes B\in\calB_0$:
$\langle A\otimes I,\,I\otimes B\rangle=\Tr(A)\Tr(B)=0$.
\end{proof}

Any Hamiltonian $H\in\End(\hilb)$ admits a unique orthogonal decomposition
\begin{equation}\label{eq:H-decomp}
  H = \underbrace{H_{\mathrm{loc}}}_{\in\,\calL}
    + \underbrace{H_{\mathrm{int}}}_{\in\,\calL^\perp},
  \qquad
  \|H\|_2^2 = \|H_{\mathrm{loc}}\|_2^2 + \|H_{\mathrm{int}}\|_2^2,
\end{equation}
where $\calL = \CC\id\oplus\calL_0$ and $H_{\mathrm{int}}=P_{\calL^\perp}(H)$.

\begin{definition}[$\eta$-stability]\label{def:stability}
A bipartition $S|E$ is \emph{$\eta$-stable} (with respect to $H$) if
\begin{equation}\label{eq:eta-stable}
  \|H_{\mathrm{int}}\|_2 \;\le\; \eta\,\|H\|_2, \qquad 0\le\eta<1.
\end{equation}
By Pythagoras, this implies $\|H_{\mathrm{loc}}\|_2\ge\sqrt{1-\eta^2}\,\|H\|_2$.
\end{definition}

%=============================================================================
\section{Quadratic Bound on Mutual Information}
\label{sec:quadratic}
%=============================================================================

\begin{theorem}[Quadratic growth of correlations]\label{thm:quadratic}
Let $\rho(0)=\rho_S\otimes\rho_E$ be a product state of full rank, and let
$\rho(t)=e^{-iHt}\rho(0)e^{iHt}$. Define
$\sigma(t):=\rho_S(t)\otimes\rho_E(t)$ and
$\Delta(t):=\rho(t)-\sigma(t)$.
Then:
\begin{equation}\label{eq:MI-quadratic}
  I(S:E)_{\rho(t)} \;=\;
  \frac{1}{2}\,\langle\Delta_1,\,\Omega_\sigma^{-1}(\Delta_1)\rangle\,t^2
  \;+\; O(t^3),
\end{equation}
where $\Delta_1:=\frac{d}{dt}\Delta(t)\big|_{t=0}$ and
$\Omega_\sigma$ is the Kubo--Mori operator
$\Omega_\sigma(X):=\int_0^1 \sigma^s\,X\,\sigma^{1-s}\,ds$.

Moreover, there exist constants $0<c_\sigma\le C_\sigma<\infty$
(depending on the spectrum of $\rho_S,\rho_E$) such that
\begin{equation}\label{eq:MI-sandwich}
  c_\sigma\,\|H_{\mathrm{int}}\|_2^2
  \;\le\;
  \frac{1}{t^2}\,I(S:E)_{\rho(t)}
  \;\le\;
  C_\sigma\,\|H_{\mathrm{int}}\|_2^2
  \;+\; O(t).
\end{equation}
\end{theorem}

\begin{proof}
The proof proceeds in four steps.

\medskip\noindent
\textbf{Step 1: Mutual information as relative entropy.}
By definition,
$I(S:E)_{\rho(t)}=D(\rho(t)\|\sigma(t))$
where $D(\rho\|\sigma)=\Tr\rho(\log\rho-\log\sigma)$.
Since $\rho(0)$ is a product state, $I(S:E)|_{t=0}=0$ and $\Delta(0)=0$.

\medskip\noindent
\textbf{Step 2: Quadratic expansion of relative entropy.}
For $\sigma$ invertible and $\|\Delta\|$ small, the standard expansion
(Hiai--Petz~\cite{HiaiPetz1991}) gives
\[
  D(\sigma+\Delta\|\sigma)
  = \tfrac{1}{2}\langle\Delta,\Omega_\sigma^{-1}(\Delta)\rangle + O(\|\Delta\|^3).
\]
Since $\Delta(t)=\Delta_1\,t+O(t^2)$, this yields~\eqref{eq:MI-quadratic}.

\medskip\noindent
\textbf{Step 3: Only $H_{\mathrm{int}}$ contributes to $\Delta_1$.}
Writing $H=H_{\mathrm{loc}}+H_{\mathrm{int}}$, we compute
$\dot\rho(0)=-i[H,\sigma]$ and
$\dot\sigma(0)=\dot\rho_S(0)\otimes\rho_E+\rho_S\otimes\dot\rho_E(0)$.
The local part $H_{\mathrm{loc}}$ generates factored evolution at first order:
$-i[H_S\otimes I+I\otimes H_E,\;\rho_S\otimes\rho_E]
= -i[H_S,\rho_S]\otimes\rho_E - i\rho_S\otimes[H_E,\rho_E]$,
which is exactly absorbed by $\dot\sigma(0)$. Therefore:
\begin{equation}\label{eq:Delta1}
  \Delta_1 = -i\Bigl(
    [H_{\mathrm{int}},\sigma]
    - \Tr_E[H_{\mathrm{int}},\sigma]\otimes\rho_E
    - \rho_S\otimes\Tr_S[H_{\mathrm{int}},\sigma]
  \Bigr)
  =: \calG_\sigma(H_{\mathrm{int}}).
\end{equation}
The map $\calG_\sigma:\calL^\perp\to\End_0(\hilb)$ is linear and depends only
on $\sigma$.

\medskip\noindent
\textbf{Step 4: Sandwich bounds.}
Since $\Omega_\sigma$ is positive definite (for $\sigma$ of full rank),
$\langle\Delta_1,\Omega_\sigma^{-1}(\Delta_1)\rangle$ is bounded above and below
by multiples of $\|\Delta_1\|_2^2$, controlled by
$\lambda_{\min}(\sigma)\le\lambda_{\max}(\sigma)$.
Similarly, $\|\calG_\sigma(H_{\mathrm{int}})\|_2$ is bounded above and below by
multiples of $\|H_{\mathrm{int}}\|_2$ (since $\calG_\sigma$ is injective on $\calL^\perp$
for generic $\sigma$).
Combining gives~\eqref{eq:MI-sandwich}.
\end{proof}

\begin{corollary}[Stability timescale]\label{cor:timescale}
For $I(S:E)_{\rho(t)}\le\varepsilon$, the timescale of stability is
\begin{equation}
  \tau \;\lesssim\; \frac{\sqrt{\varepsilon}}{\|H_{\mathrm{int}}\|_2}\times
  (\text{constant depending on }\sigma).
\end{equation}
This is a \textbf{sufficient condition} for stability, not necessary.
\end{corollary}

\begin{remark}[Pure-state regime and linear entropy]\label{rem:pure-state}
Theorem~\ref{thm:quadratic} requires a full-rank initial state $\rho(0)$,
ensuring that the Kubo--Mori operator $\Omega_\sigma$ is invertible.
In quantum circuit implementations the natural initial condition is a
pure product state $|\psi_S\rangle\otimes|\psi_E\rangle$ (rank~1), for which
$\Omega_\sigma$ degenerates.

In this regime the von~Neumann MI acquires a logarithmic correction:
since $S(\rho_S)\approx\varepsilon|\log\varepsilon|$ when the reduced state
has one eigenvalue $\varepsilon\ll 1$, the quadratic bound becomes
$I(S:E)\sim\eta^2|\log\eta|\,t^2$ rather than a clean $\eta^2 t^2$.

A more robust observable is the \emph{purity deficit}
(linear entropy)
\begin{equation}\label{eq:purity-deficit}
  \Delta(t) := 1 - \Tr\!\bigl(\rho_S(t)^2\bigr),
\end{equation}
which satisfies $\Delta(t)=O(\|H_{\mathrm{int}}\|_2^2\,t^2)$
without any logarithmic correction, even for pure initial states.
Numerical validation on a 2-qubit Heisenberg model confirms
$\Delta\propto\eta^{2.03}$ (1.3\% deviation), while the MI exponent is
$\eta^{1.78}$, consistent with the expected $|\log\eta|$ correction.
This distinction is relevant for any experimental or quantum-circuit
validation of the stability bounds.
\end{remark}

%=============================================================================
\section{Geometric Uniqueness: Two Stable Bipartitions Are Almost Parallel}
\label{sec:uniqueness-geom}
%=============================================================================

\begin{theorem}[Angle bound]\label{thm:angle}
Let $S|E$ and $S'|E'$ be two bipartitions of $\hilb$, both $\eta$-stable
with respect to $H$, with $\eta<1/2$.
Let $\theta$ denote the largest principal angle between
$\calL_0$ and $\calL_0'$.
Then:
\begin{equation}\label{eq:angle-bound}
  \sin\theta \;\le\; \frac{2\eta}{\sqrt{1-\eta^2}}.
\end{equation}
In particular, $\theta=O(\eta)$ as $\eta\to 0$.
\end{theorem}

\begin{proof}
By $\eta$-stability of $S|E$:
$\|H_{\mathrm{int}}\|_2\le\eta\|H\|_2$, hence
$\|H_{\mathrm{loc}}\|_2\ge\sqrt{1-\eta^2}\,\|H\|_2$.

Since $H_{\mathrm{loc}}\in\calL$ and $\theta$ is the angle between
$\calL_0$ and $\calL_0'$, the distance from $H_{\mathrm{loc}}$ to $\calL'$ satisfies
\[
  \mathrm{dist}(H_{\mathrm{loc}},\calL')
  \;\ge\; \sin\theta\,\|H_{\mathrm{loc}}\|_2.
\]
By the triangle inequality:
\[
  \|H_{\mathrm{int}}'\|_2
  = \mathrm{dist}(H,\calL')
  \ge \mathrm{dist}(H_{\mathrm{loc}},\calL') - \|H_{\mathrm{int}}\|_2
  \ge \sin\theta\,\|H_{\mathrm{loc}}\|_2 - \|H_{\mathrm{int}}\|_2.
\]
Imposing $\eta$-stability of $S'|E'$:
$\|H_{\mathrm{int}}'\|_2\le\eta\|H\|_2$. Therefore:
\[
  \eta\|H\|_2 \ge \sin\theta\sqrt{1-\eta^2}\,\|H\|_2 - \eta\|H\|_2,
\]
which gives $\sin\theta\le 2\eta/\sqrt{1-\eta^2}$.
\end{proof}

\begin{remark}
The identity component $\CC\id$ is shared by all bipartitions, so
$\|P_{\calL}-P_{\calL'}\|_{2\to 2}=\|P_{\calL_0}-P_{\calL_0'}\|_{2\to 2}$.
\end{remark}

%=============================================================================
\section{From Local Subspaces to Local Algebras (Lemma~2)}
\label{sec:lemma2}
%=============================================================================

\begin{lemma}[Summand identification]\label{lem:summand}
Let $\varepsilon:=\|P_{\calL_0}-P_{\calL_0'}\|_{2\to 2}<1/8$.
Suppose $d_S\neq d_E$ (so $\dim\calA_0\neq\dim\calB_0$).
Then:
\begin{equation}
  \|P_{\calA_0}-P_{\calA_0'}\|_{2\to 2} \;\le\; 4\varepsilon
  \qquad\text{and}\qquad
  \|P_{\calB_0}-P_{\calB_0'}\|_{2\to 2} \;\le\; 4\varepsilon.
\end{equation}
When $d_S=d_E$, the conclusion holds up to the possibility that
$\calA_0'\leftrightarrow\calB_0'$ (swap).
\end{lemma}

\begin{proof}
The proof uses spectral perturbation theory (Davis--Kahan).

\medskip\noindent
\textbf{Step 1: Near-isometric injection.}
For any $x\in\calA_0$ with $\|x\|_2=1$:
$\|P_{\calL_0'}x\|_2^2 = \langle x,P_{\calL_0'}x\rangle
\ge 1-\varepsilon$,
since $\langle x,(P_{\calL_0}-P_{\calL_0'})x\rangle\le\varepsilon$.

\medskip\noindent
\textbf{Step 2: Spectral separation.}
Define the self-adjoint operator
$T:=P_{\calL_0'}\,P_{\calA_0}\,P_{\calL_0'}$
acting on $\End_0(\hilb)$, which leaves $\calL_0'$ invariant.
The restriction $T|_{\calL_0'}$ has spectrum in $[0,1]$.

In the unperturbed case ($\calL_0'=\calL_0$), $T|_{\calL_0}=P_{\calA_0}|_{\calL_0}$,
which has eigenvalue~1 on $\calA_0$ (dimension $m:=d_S^2-1$) and eigenvalue~0
on $\calB_0$ (dimension $n:=d_E^2-1$). The spectral gap is~1.

The perturbation satisfies
$\|T-P_{\calA_0}\|_{2\to 2}\le 2\varepsilon$
(using $T-P_{\calA_0}=(P_{\calL_0'}-P_{\calL_0})P_{\calA_0}P_{\calL_0'}
+P_{\calL_0}P_{\calA_0}(P_{\calL_0'}-P_{\calL_0})$).

For $\varepsilon<1/8$, the spectrum of $T|_{\calL_0'}$ is separated by the
value $1/2$: eigenvalues near~1 form a cluster of multiplicity~$m$,
eigenvalues near~0 form a cluster of multiplicity~$n$.

\medskip\noindent
\textbf{Step 3: Davis--Kahan bound.}
Let $Q:=\mathbf{1}_{[1/2,1]}(T|_{\calL_0'})$ be the spectral projector
of rank~$m$. By the Davis--Kahan $\sin\Theta$ theorem~\cite{DavisKahan1970}:
\begin{equation}
  \|Q-P_{\calA_0}\|_{2\to 2}
  \;\le\; \frac{\|T-P_{\calA_0}\|_{2\to 2}}{\mathrm{gap}/2}
  \;\le\; \frac{2\varepsilon}{1/2} = 4\varepsilon.
\end{equation}

\medskip\noindent
\textbf{Step 4: Identification $Q=P_{\calA_0'}$ (non-circular).}
$Q$ is a rank-$m$ projector inside $\calL_0'=\calA_0'\oplus\calB_0'$.
Since $\|Q-P_{\calA_0}\|\le 4\varepsilon$ and $P_{\calA_0}$ has rank $m$:
\[
  \Tr(Q\,P_{\calA_0'}) = \Tr(Q\,P_{\calL_0'}) - \Tr(Q\,P_{\calB_0'})
  = m - \Tr(Q\,P_{\calB_0'}).
\]
We bound $\Tr(Q\,P_{\calB_0'})$.
Since $Q$ is $4\varepsilon$-close to $P_{\calA_0}$, any unit vector
in the range of $Q$ has component at most $4\varepsilon$ outside $P_{\calA_0}$.
Combined with $P_{\calA_0}$ being inside $\calL_0$ and $P_{\calB_0'}$ inside $\calL_0'$
(which are $\varepsilon$-close), we get
$\Tr(Q\,P_{\calB_0'})\le m\cdot C\varepsilon^2$ for small $\varepsilon$.

When $m\neq n$, $P_{\calA_0'}$ is the \emph{unique} rank-$m$ orthogonal
projector commuting with $P_{\calL_0'}$.
Since $\Tr(Q\,P_{\calA_0'})$ is close to $m$
(its maximum), and both $Q$ and $P_{\calA_0'}$ are rank-$m$ projectors
inside $\calL_0'$, it follows that $Q=P_{\calA_0'}$ for $\varepsilon$ sufficiently small.
\end{proof}

%=============================================================================
\section{From Close Algebras to Unitary Conjugation (Lemma~3)}
\label{sec:lemma3}
%=============================================================================

This section contains the most technically involved step: showing that
HS-closeness of subalgebras implies conjugation by a near-local unitary.

\begin{lemma}[Conjugation by near-local unitary]\label{lem:conjugation}
Let $\calA=\End(\hilb_S)\otimes I_E\cong M_{d_S}\otimes I_{d_E}$
and $\calA'\subset\End(\hilb)$ be another $*$-subalgebra isomorphic to $M_{d_S}$,
forming part of an alternative bipartition.
Suppose
\begin{equation}
  \delta := \|P_{\calA_0}-P_{\calA_0'}\|_{2\to 2} \;\le\; \delta_0
\end{equation}
with $\delta_0$ sufficiently small.
Then:
\begin{enumerate}[label=(\alph*)]
\item There exists a unitary $W\in U(\hilb)$ such that
  $\calA'=W\calA W^\dagger$.
\item There exist local unitaries $U_S\otimes U_E$ such that
  \begin{equation}
    \|W - (U_S\otimes U_E)\|_2 \;\le\; C(d_S,d_E)\cdot\delta.
  \end{equation}
\item If $U$ is any unitary with $\calA'=U\calA U^\dagger$, then
  \begin{equation}\label{eq:almost-local}
    U = (\tilde U_S\otimes \tilde U_E)\,(I+O(\delta))
  \end{equation}
  for some local unitaries $\tilde U_S\otimes\tilde U_E$.
\end{enumerate}
\end{lemma}

\begin{proof}
We give a constructive proof in six steps.

\medskip\noindent
\textbf{Step 1: Generalized Gell-Mann basis.}
Let $\{T_a\}_{a=1}^{d_S^2-1}$ be an orthonormal Hermitian basis of the
traceless part of $M_{d_S}$, satisfying
$\Tr(T_a^\dagger T_b)=\delta_{ab}$ and
$[T_a,T_b]=i\sum_c f_{abc}T_c$.
Define $F_a:=T_a\otimes I_E/\sqrt{d_E}$, which form an orthonormal HS basis
of $\calA_0$.

\medskip\noindent
\textbf{Step 2: Projection and controlled orthonormalization.}
Set $G_a:=P_{\calA_0'}(F_a)$. Then $\|G_a-F_a\|_2\le\delta$.
The restriction map $R:=P_{\calA_0'}|_{\calA_0}:\calA_0\to\calA_0'$ satisfies
$\|R-\mathrm{Id}\|_{2\to 2}\le\delta$, hence for $\delta<1/4$ the Gram matrix
$M=R^\dagger R$ satisfies $\|M-I_m\|\le 2\delta+\delta^2$.

Orthonormalize via $\widetilde G_a:=\sum_b(M^{-1/2})_{ba}G_b$.
Then $\{\widetilde G_a\}$ is an orthonormal HS basis of $\calA_0'$, with
$\|\widetilde G_a-F_a\|_2\le C_1\delta$ for an explicit $C_1\le 5$.

\medskip\noindent
\textbf{Step 3: Almost-multiplicativity.}
The linear map $\Phi:\calA_0\to\calA_0'$ defined by $\Phi(F_a):=\widetilde G_a$
satisfies:
\begin{equation}
  \|\Phi([F_a,F_b])-[\Phi(F_a),\Phi(F_b)]\|_2 \;\le\; C_2\delta,
\end{equation}
where $C_2$ depends on the structure constants $f_{abc}$ (hence only on $d_S$).
This extends to the full algebra: $\Phi:\calA\to\calA'$
(fixing $\Phi(\id)=\id$) satisfies
\begin{align}
  \|\Phi(XY)-\Phi(X)\Phi(Y)\|_2 &\le C_3\delta\,\|X\|_2\|Y\|_2, \\
  \|\Phi(X^*)-\Phi(X)^*\|_2 &\le C_4\delta\,\|X\|_2.
\end{align}

\medskip\noindent
\textbf{Step 4: Correction to exact $*$-homomorphism.}

This is the central technical step. We correct $\Phi$ to an exact
$*$-homomorphism using the following self-contained argument for matrix algebras.

\begin{enumerate}[label=\textbf{4.\arabic*}]
\item \textbf{Almost-idempotent correction.}
  Let $\{E_{ij}\}$ be the canonical matrix units of $M_{d_S}$ (embedded in $\calA$),
  and set $e_{ij}:=\Phi(E_{ij})$.
  The almost-multiplicativity gives
  $\|e_{ii}^2-e_{ii}\|_2\le C_3\delta$ and
  $\|e_{ii}^*-e_{ii}\|_2\le C_4\delta$.

  \emph{Spectral correction}: for a Hermitian almost-idempotent $h$ with
  $\|h^2-h\|<1/4$, the spectrum lies in
  $(-1/4,1/4)\cup(3/4,5/4)$.
  Define $p:=\mathbf{1}_{[1/2,\infty)}(h)$.
  Then $p$ is an exact orthogonal projection with
  $\|p-h\|_2\le 2\|h^2-h\|_2\le 2C_3\delta$.

  Applying this to $h_i:=\frac{1}{2}(e_{ii}+e_{ii}^*)$
  yields exact projections $p_i$ with $\|p_i-e_{ii}\|_2\le C_5\delta$.

\item \textbf{Almost-isometry correction.}
  For $i\neq j$, $e_{ij}$ satisfies
  $\|e_{ij}^*e_{ij}-p_j\|_2\le C_6\delta$ and
  $\|e_{ij}e_{ij}^*-p_i\|_2\le C_6\delta$.

  \emph{Polar correction}: define $v_{ij}$ as the partial isometry in the
  polar decomposition of $p_i\,e_{ij}\,p_j$:
  \[
    p_i\,e_{ij}\,p_j = v_{ij}\,|p_i\,e_{ij}\,p_j|.
  \]
  For $\delta$ small enough, $|p_i\,e_{ij}\,p_j|$ is invertible on $p_j\hilb$,
  so $v_{ij}$ is a well-defined partial isometry from $p_j\hilb$ to $p_i\hilb$,
  with $\|v_{ij}-e_{ij}\|_2\le C_7\delta$.

\item \textbf{Exact matrix units.}
  By construction, $\{v_{ij}\}$ satisfy the matrix unit relations exactly:
  $v_{ij}v_{kl}=\delta_{jk}v_{il}$ and $v_{ij}^*=v_{ji}$.
  This follows from the uniqueness of polar decomposition and the fact that
  the corrections are small enough that the combinatorial structure is preserved.

\item \textbf{Exact $*$-homomorphism.}
  Define $\Psi(E_{ij}):=v_{ij}$, extended linearly.
  This is an exact $*$-homomorphism $\Psi:\calA\to\End(\hilb)$ with
  $\|\Phi-\Psi\|_{2\to 2}\le C_8(d_S)\cdot\delta$.
\end{enumerate}

\medskip\noindent
\textbf{Step 5: Skolem--Noether.}
Since $\calA\cong M_{d_S}$ is simple, $\Psi$ is injective
(its kernel is a bilateral ideal, hence $\{0\}$ or $\calA$; injectivity follows
from $\|\Psi-\Phi\|$ small and $\Phi$ being injective).
Both $\calA$ and $\Psi(\calA)$ are copies of $M_{d_S}$ inside $M_d(\CC)$
with $d=d_Sd_E$. By the Skolem--Noether theorem~\cite{SkolemNoether} (for central simple algebras
in finite dimension), there exists $W\in U(d)$ such that
\begin{equation}
  \Psi(X) = WXW^\dagger \quad\forall\, X\in\calA.
\end{equation}
This proves part~(a).

\medskip\noindent
\textbf{Step 6: Near-locality via Casimir gap.}

Write $W=e^{iH_W}$ (which is valid for $W$ close to $I$ up to a local
unitary factor). Decompose $H_W$ in the HS-orthogonal decomposition
of $\End(\hilb)$ under the adjoint action of $\calA$:
\[
  H_W = H_W^{(\calA)} + H_W^{(\calB)} + H_W^{(\mathrm{int})},
\]
where $H_W^{(\calA)}\in\calA$, $H_W^{(\calB)}\in\calB$,
$H_W^{(\mathrm{int})}\in\calL^\perp$.

The condition $WF_aW^\dagger=\Psi(F_a)=F_a+O(\delta)$ implies
$[H_W,F_a]=O(\delta)$ for all generators.
Since $H_W^{(\calB)}$ commutes exactly with $\calA$, and $[H_W^{(\calA)},F_a]$
produces rotations within $\calA$, the constraint falls on $H_W^{(\mathrm{int})}$:
\[
  \|[H_W^{(\mathrm{int})},F_a]\|_2 = O(\delta) \quad\forall\,a.
\]

The adjoint action of $\calA_0\cong\mathfrak{su}(d_S)$ on
$\calL^\perp\cong\mathfrak{su}(d_S)\otimes\End_0(\hilb_E)$
has a spectral gap $\lambda_\star>0$ (the smallest nonzero eigenvalue
of the Casimir operator in the adjoint representation, which equals $d_S$
in standard normalization). Therefore:
\begin{equation}
  \|H_W^{(\mathrm{int})}\|_2 \;\le\; \frac{C_9}{\lambda_\star}\,\delta
  \;=\; \frac{C_9}{d_S}\,\delta.
\end{equation}

Setting $U_S:=e^{iH_W^{(\calA)}}|_{\hilb_S}$ and
$U_E:=e^{iH_W^{(\calB)}}|_{\hilb_E}$
(which are local unitaries), we obtain via Baker--Campbell--Hausdorff:
\[
  W = (U_S\otimes U_E)\,e^{iH_W^{(\mathrm{int})}+O(\delta^2)}
  = (U_S\otimes U_E)(I+O(\delta)).
\]
This proves part~(b) with the reformulation
$\inf_{U_S,U_E}\|W-(U_S\otimes U_E)\|_2=O(\delta)$.

For part~(c): if $U$ also satisfies $\calA'=U\calA U^\dagger$, then
$W^\dagger U\in\Norm(\calA)$.
Since the normalizer of $M_{d_S}\otimes I_{d_E}$ in $U(d)$ is
$(U(d_S)\otimes U(d_E))\cdot U(1)$ (by Skolem--Noether applied to
$\Ad_{W^\dagger U}|_\calA$), we conclude
$U=W(\tilde U_S\otimes\tilde U_E)e^{i\phi}$, hence
$U$ is almost local.
\end{proof}

%=============================================================================
\section{The Uniqueness Theorem}
\label{sec:uniqueness}
%=============================================================================

\begin{theorem}[Uniqueness of stable bipartition]\label{thm:uniqueness}
Let $H$ be a Hamiltonian on $\hilb$ with $\|H\|_2>0$.
Suppose there exist two bipartitions $S|E$ and $S'|E'$, both
$\eta$-stable with $\eta$ sufficiently small, and with
$d_S\neq d_E$ (no swap ambiguity).
Assume no block decomposition (i.e., $\calA$ is a simple factor).
Then:
\begin{enumerate}[label=(\roman*)]
\item The principal angle satisfies $\theta(\calL_0,\calL_0')=O(\eta)$.
\item The algebra distance satisfies $\|P_{\calA_0}-P_{\calA_0'}\|_{2\to 2}=O(\eta)$.
\item The unitary $U$ connecting the bipartitions is almost local:
  \begin{equation}
    U = (\tilde U_S\otimes\tilde U_E)(I+O(\eta)).
  \end{equation}
\item In particular, any bipartition that is genuinely nonlocal
  (i.e., not related to $S|E$ by a local unitary) cannot be $\eta$-stable
  for small $\eta$.
\end{enumerate}
\end{theorem}

\begin{proof}
Combine Theorem~\ref{thm:angle} ($\theta=O(\eta)$),
Lemma~\ref{lem:summand} ($\|P_{\calA_0}-P_{\calA_0'}\|\le 4\varepsilon$
with $\varepsilon=O(\eta)$),
and Lemma~\ref{lem:conjugation} ($U$ almost local from $\delta=O(\eta)$).
\end{proof}

\begin{remark}[Exceptional cases]\label{rem:exceptions}
Three types of degeneracy can occur:
\begin{enumerate}[label=(\alph*)]
\item \textbf{Swap ($d_S=d_E$):} the roles of $S$ and $E$ can be interchanged.
  Uniqueness holds up to this exchange.
\item \textbf{Block structure (superselection):} if $H$ decomposes as
  $H=\bigoplus_\alpha H^{(\alpha)}$ on $\hilb=\bigoplus_\alpha\hilb^{(\alpha)}$,
  stable factorizations may exist independently within each sector
  (``virtual subsystems'' in the sense of Zanardi--Knill~\cite{ZanardiLidarLloyd2004}).
\item \textbf{Symmetries:} continuous symmetries of $H$ may allow families
  of equivalent factorizations related by the symmetry group.
\end{enumerate}
\end{remark}

%=============================================================================
\section{Variational Principle}
\label{sec:variational}
%=============================================================================

\begin{definition}[Interaction functional]
For a Hamiltonian $H$ on $\hilb$ and a bipartition $S|E$:
\begin{equation}
  \calF(S|E) := \|P_{\calL^\perp}(H)\|_2^2 = \|H_{\mathrm{int}}\|_2^2.
\end{equation}
\end{definition}

\begin{theorem}[Variational selection of physical factorization]
\label{thm:variational}
Under the hypotheses of Section~\ref{sec:setup}
(finite-dimensional, ideal clock $H_{\mathrm{tot}}=H_C+H_{SE}$):
\begin{enumerate}[label=(\roman*)]
\item \textbf{Physical meaning}: $\calF(S|E)$ controls the quadratic growth
  of mutual information via the sandwich bounds~\eqref{eq:MI-sandwich}.
  Minimizing $\calF$ minimizes the rate of correlation creation.

\item \textbf{Formal equivalence}: for any full-rank product state
  $\sigma=\rho_S\otimes\rho_E$ relative to the bipartition $S|E$,
  \begin{equation}
    c_\sigma\,\calF(S|E) \;\le\;
    \left.\frac{d^2}{dt^2}I(S:E)_{\rho(t)}\right|_{t=0}
    \;\le\; C_\sigma\,\calF(S|E).
  \end{equation}

\item \textbf{Robustness of the minimizer}: by the uniqueness theorem
  (Theorem~\ref{thm:uniqueness}), any bipartition not almost-locally
  equivalent to the minimizer $(S|E)_\star$ has
  $\calF=O(\|H\|_2^2)$, while $\calF((S|E)_\star)=O(\eta^2\|H\|_2^2)$.
  Since the constants $c_\sigma,C_\sigma$ vary continuously and are bounded
  in any family of states with spectrum bounded away from zero,
  the gap in $\calF$ cannot be compensated by variation in constants.
  Hence:
  \begin{equation}
    \arg\min_{S|E}\calF(S|E)
    \;=\;
    \arg\min_{S|E}\left.\frac{d^2}{dt^2}I(S:E)(t)\right|_{t=0}
  \end{equation}
  up to almost-local equivalences (and swap when $d_S=d_E$, and block degeneracies).
\end{enumerate}
\end{theorem}

\begin{remark}
The variational principle does not invoke an ensemble or averaging procedure.
The functional $\calF$ is purely algebraic (depends only on $H$ and the
bipartition), and its physical interpretation via mutual information
curvature is justified by the bilateral bounds that hold for every
full-rank product state.
For pure initial states, $\calF$ equally controls the purity deficit
$\Delta=1-\Tr(\rho_S^2)$ (see Remark~\ref{rem:pure-state}), extending
the variational characterization to the experimentally relevant regime.
\end{remark}

%=============================================================================
\section{Effective Dynamics of $\rho_S(t)$ (Connection to Page--Wootters)}
\label{sec:dynamics}
%=============================================================================

We now connect the stability analysis to the Page--Wootters formula.

\begin{theorem}[Controlled effective dynamics]\label{thm:dynamics}
Under the ideal clock hypothesis ($H_{\mathrm{tot}}=H_C+H_{SE}$,
$H_{\mathrm{tot}}|\Psi\rangle=0$),
using the optimal bipartition $(S|E)_\star$ with
$\|H_{\mathrm{int}}\|_2\le\eta\|H_{SE}\|_2$,
the reduced state
$\rho_S(t)=\Tr_E[\rho_{SE}(t)]$ satisfies:

In the interaction picture with respect to $H_0:=H_S\otimes I+I\otimes H_E$,
to second order in $H_{\mathrm{int}}$:
\begin{equation}\label{eq:effective}
  \dot\rho_S(t) = -i[H_S+\bar H_S,\;\rho_S(t)]
  - \int_0^t \Tr_E\!\bigl[H_{\mathrm{int}}^I(t),
    [H_{\mathrm{int}}^I(s),\rho_S(t)\otimes\rho_E]\bigr]\,ds
  + O(\eta^3 t^2),
\end{equation}
where $\bar H_S:=\Tr_E(H_{\mathrm{int}}\,\rho_E)$ is the mean-field correction and
$H_{\mathrm{int}}^I(t):=e^{iH_0t}H_{\mathrm{int}}e^{-iH_0t}$.
\end{theorem}

\begin{proof}[Proof sketch]
Start from the exact equation in interaction picture:
$\dot\rho_{SE}^I(t)=-i[H_{\mathrm{int}}^I(t),\rho_{SE}^I(t)]$.
Iterate once (Dyson to second order) and trace over $E$.

The first-order term gives the mean-field unitary correction $\bar H_S$.
The second-order term is the dissipative kernel.
We substitute $\rho_{SE}(0)=\rho_S(0)\otimes\rho_E+O(\eta t)$
in the second-order term (justified by the quadratic bound on correlations:
$I(S:E)(t)=O(\eta^2 t^2)$, so the state remains approximately product
at the order we are working). The error is $O(\eta^3 t^3)$.
\end{proof}

\begin{corollary}[Almost-unitary evolution]
The dissipative (non-unitary) contribution to $\dot\rho_S$ is of order
$\eta^2\|H_{SE}\|_2^2\cdot t$.
In the optimal bipartition ($\eta$ small), the evolution is almost unitary:
\begin{equation}
  \rho_S(t) \approx e^{-i(H_S+\bar H_S)t}\,\rho_S(0)\,e^{i(H_S+\bar H_S)t}
  + O(\eta^2 t^2).
\end{equation}
In a non-optimal bipartition ($\|H_{\mathrm{int}}\|_2\sim\|H\|_2$),
the dissipation is of order~1 and the subsystem description breaks down rapidly.
\end{corollary}

\begin{remark}[Structure of the dissipative term]
Using the operator Schmidt decomposition $H_{\mathrm{int}}=\sum_\alpha A_\alpha\otimes B_\alpha$,
the second-order kernel takes the Kossakowski--Lindblad form:
\begin{equation}
  \sum_{\alpha,\beta} C_{\alpha\beta}(t,s)\left(
    A_\alpha^I(t)\,\rho_S\,A_\beta^I(s)^\dagger
    - \tfrac{1}{2}\{A_\beta^I(s)^\dagger A_\alpha^I(t),\rho_S\}
  \right),
\end{equation}
with bath correlation functions
$C_{\alpha\beta}(t,s)=\Tr(B_\alpha^I(t)\rho_E\,B_\beta^I(s)^\dagger)$.
All coefficients are $O(\|H_{\mathrm{int}}\|^2)=O(\eta^2\|H_{SE}\|_2^2)$.
\end{remark}

%=============================================================================
\section{Summary: The Complete Chain}
\label{sec:summary}
%=============================================================================

The results assemble into a single logical chain, starting from the
Page--Wootters formula~\eqref{eq:pw-formula}:

\begin{center}
\renewcommand{\arraystretch}{1.4}
\begin{tabular}{r@{\;$\Longrightarrow$\;}l@{\qquad}l}
  PW formula & presupposes $S|E$ & (Section~\ref{sec:setup}) \\
  Stability of $S|E$ & $I(S:E)(t)\le Kt^2\|H_{\mathrm{int}}\|_2^2$ &
    (Theorem~\ref{thm:quadratic}) \\
  $\|H_{\mathrm{int}}\|_2\le\eta\|H\|_2$ & bipartition is $\eta$-stable &
    (Definition~\ref{def:stability}) \\
  Two $\eta$-stable bipartitions & $\theta=O(\eta)$ &
    (Theorem~\ref{thm:angle}) \\
  $\theta=O(\eta)$ & $\|P_{\calA_0}-P_{\calA_0'}\|=O(\eta)$ &
    (Lemma~\ref{lem:summand}) \\
  Close algebras & $U\approx U_S\otimes U_E$ &
    (Lemma~\ref{lem:conjugation}) \\
  Uniqueness & $\calF(S|E)$ is variational principle &
    (Theorem~\ref{thm:variational}) \\
  Optimal bipartition & almost-unitary $\rho_S(t)$ &
    (Theorem~\ref{thm:dynamics})
\end{tabular}
\end{center}

\bigskip
\noindent\textbf{Conclusion.}
The partial trace $\Tr_E$ in the Page--Wootters formula is not an arbitrary
choice. It is anchored by the Hamiltonian's structure:
the physically meaningful factorization is the one that minimizes the
HS distance of $H$ to the local subspace, which equivalently minimizes
the initial rate of correlation creation and maximizes the unitarity of the
effective subsystem dynamics.
This factorization is essentially unique (up to local unitaries,
swap when $d_S=d_E$, and block decompositions from symmetries).

\subsection*{Open directions}
\begin{itemize}[nosep]
\item Extension to non-ideal clocks ($H_{C\text{-}SE}\neq 0$):
  how clock quality interacts with factorization stability.
\item Generalization from bipartition to multipartite factorization
  ($\bigotimes_i\hilb_i$): spatial locality emergence.
\item Explicit computation of $\calF$ for physically motivated models
  (spin chains, lattice systems).
\item Connection to decoherence theory and einselection~\cite{Zurek2003}:
  the pointer basis as the ``$S$-internal'' basis selected by the
  optimal factorization.
\item Numerical and quantum-circuit validation: the quadratic scaling
  of the purity deficit has been verified on a 2-qubit Heisenberg
  model both via exact diagonalization and via Trotterized circuits
  on an IBM Quantum simulator (Qiskit \texttt{StatevectorEstimator}),
  confirming $\Delta\propto\eta^{2.03}$ with sub-2\% error.
  Extension to real quantum hardware and larger system sizes
  is a natural next step.
\end{itemize}

The results presented here form the mathematical backbone of the
Page--Wootters--Assemblage framework developed in the companion
paper~\cite{Giani2026}.

%=============================================================================
\section{Table of Constants}
\label{sec:constants}
%=============================================================================

For reference, we collect the explicit dependencies of the constants
appearing throughout.

\begin{center}
\renewcommand{\arraystretch}{1.3}
\begin{tabular}{lll}
\hline
\textbf{Constant} & \textbf{Appears in} & \textbf{Depends on} \\
\hline
$K,c_\sigma,C_\sigma$ & Quadratic bound (Thm.~\ref{thm:quadratic})
  & Spectrum of $\rho_S,\rho_E$ \\
$4$ (as in $4\varepsilon$) & Lemma~\ref{lem:summand}
  & None (universal) \\
$C_1\le 5$ & Orthonormalization (Lem.~\ref{lem:conjugation}, Step 2)
  & For $\delta<1/4$ \\
$C_2,C_3$ & Almost-multiplicativity (Step 3)
  & $d_S$ (structure constants of $\mathfrak{su}(d_S)$) \\
$C_5,\ldots,C_8$ & Idempotent/isometry correction (Step 4)
  & $d_S$ \\
$\lambda_\star=d_S$ & Casimir gap (Step 6)
  & $d_S$ (representation theory) \\
$C(d_S,d_E)$ & Final constant (Lem.~\ref{lem:conjugation})
  & Polynomial in $d_S$; independent of $d_E$ \\
\hline
\end{tabular}
\end{center}

\bigskip
Notably, the geometric core of the argument (Theorem~\ref{thm:angle}
and the factor $4\varepsilon$ in Lemma~\ref{lem:summand}) is
\emph{dimension-independent}, while the algebraic correction
(Lemma~\ref{lem:conjugation}) introduces polynomial dependence on $d_S$
through the structure of $\mathfrak{su}(d_S)$.

\begin{thebibliography}{99}

\bibitem{PageWootters1983}
D.~N.~Page and W.~K.~Wootters,
``Evolution without evolution: Dynamics described by stationary observables,''
\emph{Phys.\ Rev.\ D} \textbf{27}, 2885 (1983).

\bibitem{Wootters1984}
W.~K.~Wootters,
``Time replaced by quantum correlations,''
\emph{Int.\ J.\ Theor.\ Phys.} \textbf{23}, 701 (1984).

\bibitem{HiaiPetz1991}
F.~Hiai and D.~Petz,
``The proper formula for relative entropy and its asymptotics
in quantum probability,''
\emph{Commun.\ Math.\ Phys.} \textbf{143}, 99 (1991).

\bibitem{DavisKahan1970}
C.~Davis and W.~M.~Kahan,
``The rotation of eigenvectors by a perturbation.~III,''
\emph{SIAM J.\ Numer.\ Anal.} \textbf{7}, 1 (1970).

\bibitem{SkolemNoether}
R.~S.~Pierce,
\emph{Associative Algebras},
Grad.\ Texts in Math.\ Vol.~88 (Springer, New York, 1982),
Chap.~12 (Skolem--Noether theorem).

\bibitem{ZanardiLidarLloyd2004}
P.~Zanardi, D.~A.~Lidar, and S.~Lloyd,
``Quantum tensor product structures are observable induced,''
\emph{Phys.\ Rev.\ Lett.} \textbf{92}, 060402 (2004).

\bibitem{Zurek2003}
W.~H.~Zurek,
``Decoherence, einselection, and the quantum origins of the classical,''
\emph{Rev.\ Mod.\ Phys.} \textbf{75}, 715 (2003).

\bibitem{Giani2026}
G.~Giani,
``The Page--Wootters--Assemblage framework: Observer-free quantum
mechanics from relational structure,''
(2026), companion paper.

\end{thebibliography}

\end{document}
