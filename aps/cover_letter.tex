\documentclass[12pt]{letter}
\usepackage[margin=1in]{geometry}
\usepackage{amsmath,amssymb}
\usepackage{hyperref}

\signature{Gabriel Giani Moreno\\Independent Researcher}
\address{}
\date{\today}

\begin{document}

\begin{letter}{Editors\\Physical Review A\\American Physical Society}

\opening{Dear Editors,}

Please find attached our manuscript entitled:

\bigskip
\begin{center}
\textit{``The Observer as a Local Breakdown of Atemporality:\\
Relational Time and an Informational Arrow from Quantum Clocks.''}
\end{center}
\bigskip

In this work, we present a unified operational framework addressing the problem of time by showing that three traditionally distinct aspects---(i)~relational Schr\"odinger dynamics, (ii)~the thermodynamic arrow of time, and (iii)~observer-dependent temporal parameterization---emerge from a single conditional construction applied to a globally stationary (atemporal) state.

The central object of the paper is the conditional reduced state
%
\begin{equation*}
\rho_S(t)=\frac{\mathrm{Tr}_E\!\left[\langle t|_C\,|\Psi\rangle\langle\Psi|\,|t\rangle_C\right]}{p(t)},
\end{equation*}
%
which we demonstrate generates all three pillars from one operational pipeline:
\begin{itemize}
\item projection onto a physical clock subsystem,
\item partial trace over inaccessible degrees of freedom,
\item locality of the clock as a subsystem rather than an external parameter.
\end{itemize}

The manuscript makes the following concrete contributions:

\begin{enumerate}
\item \textbf{Unified operational synthesis.}
While Page--Wootters conditioning, informational arrows, and temporal quantum reference frames have been studied separately, we show explicitly that they are three readings of the same conditional structure.

\item \textbf{Clock-orientation covariance theorem.}
We prove exact covariance under arbitrary relabelings of the clock basis, including full reversal, and demonstrate numerically that reversal deterministically inverts the entropy arrow within the same formalism.

\item \textbf{Continuum limit and emergent clock transformation group.}
We show uniform convergence as clock resolution increases and demonstrate that inter-clock transformations approach an affine time-reparameterization structure in the large-clock limit.

\item \textbf{Robustness analysis.}
We perform gravitational, structural, and necessity tests, explicitly identifying which postulate violations destroy which pillar.

\item \textbf{Experimental validation on IBM Quantum hardware.}
Both relational dynamics (Pillar~1) and the informational arrow (Pillar~2) are implemented on superconducting qubits (\texttt{ibm\_torino}), with quantified noise characterization and reproducible agreement with theory.
\end{enumerate}

The work is theoretical in structure but includes explicit hardware implementation and device-level noise analysis. It is therefore positioned at the intersection of quantum foundations, quantum information, and operational approaches to time in finite systems.

We believe the manuscript is appropriate for \textit{Physical Review~A}, as it addresses fundamental questions in quantum mechanics using finite-dimensional systems, open-system analysis, and experimentally testable constructions.

The manuscript is original, has not been published elsewhere, and is not under consideration by another journal.

\closing{Thank you for your consideration.\\ \\Sincerely,}

\end{letter}
\end{document}
